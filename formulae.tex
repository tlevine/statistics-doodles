\documentclass{article}
\usepackage{amsmath}
\usepackage{hyperref}
\setlength\parindent{0pt}
\setlength\parskip{0.5em}
\title{Formul\ae\ for concepts that Tom is drawing}
\author{Thomas Levine}

\begin{document}
\maketitle

These formul\ae\ are for the bivariate (rather than more-than-two) populations (rather than samples).

\section{Covariance}
Conceptually, it's this.
$$ \sigma_{xy} = \operatorname{E}{\big[\left(x - \operatorname{E}[x])(y - \operatorname{E}[y]\right)\big]} $$

More precisely, it could be this.
$$ \sigma_{xy} = \frac{1}{N}\sum_{i=1}^{N}\left( x_{i}-\bar{x} \right) \left( y_{i}-\bar{y} \right) $$

That's the sum ($\sum_{i=1}^{N}$) of rectangles
($\left( x_{i}-\bar{x} \right) \left( y_{i}-\bar{y} \right)$)
divided by the number of observations/rectangles ($\frac{1}{N}$).

As matrix arithmetic, it's written in a way that works for more than two variables.
($\Sigma$ is the covariance matrix.)
$$ \Sigma=\mathrm{E}
\left[
 \left(
  \textbf{X} - \mathrm{E}[\textbf{X}]
   \right)
  \left(
   \textbf{X} - \mathrm{E}[\textbf{X}]
    \right)^{\rm T}
    \right]
$$

It's multiplication of many $\textbf{X}$ by many $\textbf{X}$, and that's lots of
rectangles for lots of covariances.

In R, it's this.

\begin{verbatim}
# Two variables
cov(x,y)

# More variables
var(iris[-5])
cov(iris[-5])
\end{verbatim}

\section{Variance}
Variance is the covariance of something with itself.
$$ \sigma_{x}^2 = \sigma_{xx} = \operatorname{E}{\big[\left(x - \operatorname{E}[x]\right)^2\big]} $$

More precisely, like the covariance, it could be this.
$$ \sigma_{x}^2 = \frac{1}{N}\sum_{i=1}^{N}\left( x_{i}-\bar{x} \right) \left( x_{i}-\bar{x} \right) $$

And that reduces to this.
$$ \sigma_{xy} = \frac{1}{N}\sum_{i=1}^{N}\left( x_{i}-\bar{x} \right)^2 $$

This time, the top is a sum of squares ($\left( x_{i}-\bar{x} \right)^2$).

As matrix arithmetic, it's the diagonal of the covariance because those are the cells
where we're multiplying the the same \textbf{x}.
$$ diag\left(\Sigma\right) $$

In R, all of these would work.
\begin{verbatim}
# One variable
var(x)
cov(x,x)

# More variables
lapply(iris[-5],var)
diag(cov(iris[-5]))
\end{verbatim}

\section{Correlation}
Correlation (specifically, the Pearson product-moment correlation coefficient---there are others)
measures how much the variables linearlly depend on each other.

For perfectly linear variables that move together, $\rho_{xy}$ will be $1$.
For perfectly linear variables that move oppositely, $\rho_{xy}$ will be $-1$.

$\rho_{xy}$ is just the covariance ($\sigma_{xy}$) divided by
the product of the standard deviations ($\sigma_x\sigma_y$).

$$ \rho_{xy} = \frac{\sigma_{xy}}{\sigma_x\sigma_y} $$

We can also factor out the $N$ term and write it as the sum of the rectangles divided by
the rectangle formed by the square roots of the sums of the squares.
$$ \rho = \frac
{ \sum_{i=1}^{N}\left( x_{i}-\bar{x} \right) \left( y_{i}-\bar{y} \right) }
{
  \sqrt{\sum_{i=1}^{N}\left( x_{i}-\bar{x} \right)^2}
  \sqrt{\sum_{i=1}^{N}\left( y_{i}-\bar{y} \right)^2}
}
$$

If you need to be convinced that $\sigma_{xy}$ is no greater than $\sigma_x\sigma_y$,
recall that the covariance includes negative rectangles and the variance does not;

\section{Ordinary least-squares regression}
Simple regression is a best fit line between two variables; we are looking for $\alpha$ and $\beta$.
$$ \hat{y}_i = \alpha + \beta x_i$$

$\beta$ is just the covariance divided by the variance.
$$ \beta = \sigma{xy}/\sigma_x^2 $$

We can also factor out the $N$ term and write it as the sum of the rectangles divided by the sum of the squares.
$$ \beta = \frac
{\frac{1}{N}\sum_{i=1}^{N}\left( x_{i}-\bar{x} \right) \left( y_{i}-\bar{y} \right)}
{\frac{1}{N}\sum_{i=1}^{N}\left( x_{i}-\bar{x} \right)^2}
$$
$$ \beta = \frac
{\sum_{i=1}^{N}\left( x_{i}-\bar{x} \right) \left( y_{i}-\bar{y} \right)}
{\sum_{i=1}^{N}\left( x_{i}-\bar{x} \right)^2}
$$

In matrix form, we might right this.
$$B = \left(X^TX\right)^{-1}X^Ty$$

The part on the right ($X^Ty$) is multiplication of x-values by y-values,
so it's the sum of rectangles like in covariance.

We invert the part on the left $\left(\left(X^TX\right)^\textbf{-1}\right)$,
so it's sort of like a denominator. We're multiplying x-values by
x-values, so it's a sum of squares, like in variance.

\section{Credits}
Some formul\ae\ were lifted from these pages.
\begin{itemize}
\item \url{http://en.wikipedia.org/w/index.php?title=Covariance&action=edit}
\item \url{http://en.wikipedia.org/w/index.php?title=Pearson_product-moment_correlation_coefficient&action=edit&section=3}
\item \url{http://en.wikipedia.org/w/index.php?title=Ordinary_least_squares&action=edit}
\end{itemize}
\end{document}
